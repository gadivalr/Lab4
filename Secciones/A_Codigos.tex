\section{Códigos}
\begin{listing}[H]
  \scriptsize
  \begin{minted}{cpp}
    // Inclusión de las librerías necesarias para la comunicación y los sensores.
#include <Wire.h>     // Para la comunicación I2C (usada por el sensor BH1750).
#include <DHT.h>      // Para el sensor de temperatura y humedad DHT11.
#include <BH1750.h>   // Para el sensor de luz ambiental GY-30 (BH1750).
#include <Nextion.h>  // Para comunicarse con la pantalla Nextion.

// Definiciones para el sensor DHT11.
#define DHTTYPE DHT11   // Se especifica que el tipo de sensor es DHT11.
#define DHTPIN 9        // Se define que el pin de datos del DHT11 está conectado al pin digital 9 del Arduino.

// Creación de los objetos para los sensores.
DHT dht(DHTPIN, DHTTYPE); // Se inicializa el objeto 'dht' con el pin y el tipo definidos.
BH1750 GY30;               // Se inicializa el objeto 'GY30' para el sensor de luz.

// La función setup() se ejecuta una sola vez cuando el Arduino se enciende o se resetea.
void setup() {
  // Inicializa la comunicación serial a 9600 baudios. Es la velocidad a la que se comunicará con la pantalla Nextion.
  Serial.begin(9600);
  // Inicializa el sensor DHT.
  dht.begin();

  // Inicializa la comunicación con la pantalla Nextion (esta función podría venir de la librería Nextion).
  nexInit();
  // Inicializa el bus de comunicación I2C.
  Wire.begin();
  // Inicializa el sensor de luz BH1750.
  GY30.begin();
}

// La función loop() se ejecuta repetidamente después de que setup() ha terminado.
void loop() {
  // Pequeña pausa de 100 milisegundos para estabilizar las lecturas y no saturar los sensores.
  delay(100);
  // Lee los valores de los sensores y los guarda en variables.
  float hum1 = dht.readHumidity();      // Lee la humedad relativa (en %).
  float temp1 = dht.readTemperature();  // Lee la temperatura (en °C).
  int lux1 = GY30.readLightLevel();     // Lee el nivel de luz (en lux).

  // Convierte los valores numéricos a tipo String para poder enviarlos a la pantalla.
  String tempString = String(temp1, 1); // Convierte la temperatura a String, con 1 decimal.
  String humString = String(hum1, 0);   // Convierte la humedad a String, sin decimales.
  String luxString = String(lux1);      // Convierte el valor de lux a String.

  // --- Envío de datos a los campos de texto de la pantalla Nextion ---

  // Comando para actualizar el campo de texto "temp.txt" en la pantalla.
  Serial.print("temp.txt=\"" + tempString); // Envía la primera parte del comando: temp.txt="24.5
  Serial.write(176); // Envía el código ASCII del símbolo de grado '°'.
  Serial.print("C\"");                       // Envía la 'C' y las comillas de cierre.
  terminarComando();                       // Envía los tres bytes 0xFF para finalizar el comando.

  // Comando para actualizar el campo de texto "hum.txt".
  enviarComando("hum.txt=\"" + humString + "%\""); // Usa la función helper para enviar "hum.txt="50%""

  // Comando para actualizar el campo de texto "lux.txt".
  enviarComando("lux.txt=\"" + luxString + " lx\""); // Usa la función helper para enviar "lux.txt="1200 lx""

  // --- Envío de datos a los gráficos (Waveform) de la pantalla Nextion ---

  // Actualiza el gráfico de luminosidad (lux).
  NexWaveform lumwave = NexWaveform(0, 4, "lumwave"); // Declara un objeto para el gráfico con ID de página 0, ID de componente 4, y nombre "lumwave".
  int plotlux = map(lux1, 0, 20000, 0, 255);          // Mapea el valor de lux (0-20000) a la escala del gráfico (0-255).
\end{minted}
   \end{listing}

\begin{listing}[H]
  \scriptsize
  \begin{minted}{cpp}
   plotlux = constrain(plotlux, 0, 255);   // Se asegura que el valor no se salga del rango 0-255.
  lumwave.addValue(0, plotlux);                      // Añade el nuevo valor al canal 0 del gráfico.

  // Actualiza el gráfico de temperatura.
  float temp10 = temp1 * 10.0;                       // Multiplica por 10 para trabajar con un entero y no perder el primer decimal.
  int tempint = (int)temp10;                         // Convierte el valor a entero (ej: 24.5 -> 245).
  NexWaveform tempwave = NexWaveform(0, 6, "tempwave"); // Declara el objeto para el gráfico de temperatura.
  int plottemp = map(tempint, 0, 400, 0, 255);         // Mapea la temperatura (0.0-40.0 -> 0-400) a la escala del gráfico (0-255).
  plottemp = constrain(plottemp, 0, 255);              // Asegura que el valor esté en el rango correcto.
  tempwave.addValue(0, plottemp);                      // Añade el valor al gráfico.

  // Actualiza el gráfico de humedad.
  NexWaveform humwave = NexWaveform(0, 7, "humwave");  // Declara el objeto para el gráfico de humedad.
  int plothum = map(hum1, 0, 100, 0, 255);             // Mapea la humedad (0-100%) a la escala del gráfico (0-255).
  plothum = constrain(plothum, 0, 255);                // Asegura que el valor esté en el rango correcto.
  humwave.addValue(0, plothum);                        // Añade el valor al gráfico.
}

// --- Funciones auxiliares (helpers) ---

// Función para enviar un comando completo a la pantalla Nextion de una manera más limpia.
void enviarComando(String cmd) {
  Serial.print(cmd);     // Imprime la cadena de texto del comando.
  terminarComando();     // Llama a la función que añade los bytes de finalización.
}

// Función que envía la secuencia de terminación de comando requerida por Nextion.
void terminarComando() {
  Serial.write(0xFF);    // Envía el primer byte de terminación.
  Serial.write(0xFF);    // Envía el segundo byte de terminación.
  Serial.write(0xFF);    // Envía el tercer byte de terminación.
}


  \end{minted}
  \caption{Codigo Implementado}
  \label{lst:cod-1}
\end{listing}
